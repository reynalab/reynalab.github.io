\documentclass[12pt]{article}

\usepackage[utf8]{inputenc}
\usepackage{mathtools, parskip, fullpage}

\newcommand{\hzline}{\bigskip\hrule\medskip}

\begin{document}

Name: \line(1,0){195}\\

\begin{center}
\textbf{Quiz 6}\\
CS584: Deep Learning\\
Spring 2020
\end{center}

\hzline

Please submit this quiz by Wednesday, April 15 to the submission form on the course website. Each answer should have a few (or more) thoughtful, complete sentences.

\hzline

\textbf{Problem 1.} (\emph{4 points})
What does the word ``adversarial'' mean in ``generative adversarial networks'' and ``adversarial examples''? Does ``adversarial'' mean the same thing or a different thing in these terms?

\vspace{1.67in}

\textbf{Problem 2.} (\emph{3 points})
Are generative adversarial networks essentially learning the solution to a multi-objective optimization problem? If so, then how do GANs fit into this framework? If not, then what are GANs doing instead of this?

\vspace{1.67in}

\textbf{Problem 3.} (\emph{3 points})
Adversarial examples may raise concerns about privacy and security, but they also illustrate the robustness of machine learning models. What can we learn about model robustness from adversarial examples?

\vspace{1.67in}

\end{document}